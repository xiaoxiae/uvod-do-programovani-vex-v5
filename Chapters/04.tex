\documentclass[../main.tex]{subfiles}


\begin{document}
	\section{Teorie autonomního ovládání}

	Tato část je (oproti ostatním) více teoretická, nicméně obsahuje informace důležité k programování autonomního robota. Autonomním robotem tu rozumíme robota, který se pohybuje zcela sám (na základě programu, který vykonává), a ne např. člověkem s ovladačem.

	K autonomnímu ovládání robota existuje řada metod, které se liší převážně přesností. Ta jde však  ruku v ruce se složitostí provedení a proto se budeme bavit pouze o těch nejzákladnějších metodách.

	\subsection{Mrtvé počítání}
	Mrtvé počítání je nejzákladnější typ autonomního ovládání robota a funguje na velice jednoduchém principu: pokud chceme aby robot jel \si{20m} a víme, že se pohybuje průměrnou rychlostí \si{5m/s}, tak musí jet $20 / 5 = \si{4s}$.

	Princip je to velice jednoduchý, ale poněkud nepřesný -- co se stane, když:
	\begin{itemize}
		\item začne docházet baterie a robot pojede pomaleji?
		\item robotu proklouzne kolečko a na chvilku se zastaví?
	\end{itemize}

	Tyto problémy nejdou metodou mrtvého počítání řešit, protože robot neví, co se okolo něho děje. Nevyužívá totiž žádné senzory, které by ho informovali o tom, že se něco změnilo/pokazilo.

	\begin{question}
		Několikrát změřte průměrnou rychlost našeho robota, když motory pustíte na maximální výkon po dobu \si{5s}. Jak moc jsou výsledky přesné (liší se od sebe o milimetry/centimetry)?
	\end{question}

	\begin{question}
		Napište program v jazyce Blocky, po jehož spuštění robot ujede \si{3m}. Použijte k tomu jeho průměrnou rychlost vypočítanou v minulém příkladu.
	\end{question}

	\subsection{Bang-bang}
	Bang-bang se snaží do jisté míry napravit nedostatky minulé metody tím, že do ovládání zakomponuje data o tom, co se děje. Z kapitoly \ref{cha:encoder} víme, že každý z našich motorů obsahuje enkóder a můžeme tedy měřit, kolik otáček dané robotovo kolo udělalo.

	Když nyní chceme, aby robot ujel nějakou vzdálenost, tak nejprve spočítáme obvod kolečka vzorečkem
	$$\text{obvod} = 2 \cdot \pi \cdot \text{poloměr}$$
	a poté počet otáčky, které musí robot udělat, vzorečkem
	$$\text{otáčky} = \frac{\text{vzdálenost}}{\text{obvod}} =  \frac{\text{vzdálenost}}{2 \cdot \pi \cdot \text{poloměr}}$$ 
	Samotný princip bang-bangu je ten, že zapneme motor a kontrolujeme, jakou vzdálenost už robot ujel~--~pokud už jsme přesáhli cíl, tak robota zastavíme. Přesně to jsme vlastně dělali v kapitole \ref{cha:distanceride}.

	Díky zakomponování dat z reálného světa jsme sice vyřešili problémy jako prokluzování koleček a vybitá baterie, ale stále máme problém s přesností -- i pokud zastavíme hned po přesažení cíle, tak robot díky setrvačnosti tento cíl o nějakou (často i nemalou) vzdálenost přejede.

	Tento problém skvěle řeší pokročilejší metody, kde jedna z nejpoužívanějších je PID. Ta počítá, jak se robot pohyboval v minulosti, jak se pohybuje aktuálně a jak se pravděpodobně bude pohybovat v budoucnosti a je díky tomu (pokud správně naprogramovaná) výrazně přesnější\footnote{Dobrá ukázka PIDu je např. toto video: \href{https://www.youtube.com/watch?v=4Y7zG48uHRo}{https://www.youtube.com/watch?v=4Y7zG48uHRo}.}. 

	\begin{question}
		Spočítejte poloměr koleček na našem robotovi. Kolik metrů ujede za deset otáček?
	\end{question}

	\begin{solution}
		Poloměr kolečka můžeme změřit například pravítkem. Čím přesnější budeme, tím přesněji bude robot jezdit a tím snazší ho bude programovat.

		Počet otáček spočítáme tím, že přenásobíme obvod ($2 \cdot \pi \cdot \text{poloměr}$) číslem $10$.
	\end{solution}

	\begin{question}
		Napište program v jazyce Blocky, po jehož spuštění robot ujede přesně \si{3m}. Použijte vzoreček výše a poloměr kolečka spočítaný v minulém příkladu.
	\end{question}

	\begin{question*}
		Napište program v jazyce Blocky, po jehož spuštění se robot otočí o přesně 90\degree. Nezkoušejte tipovat, výsledek spočítejte přesně.
	\end{question*}

	\begin{solution*}
		K tomu, abychom mohli přesně spočítat, o kolik se má robot otočit, budeme potřebovat vědět, jaká je vzájemná vzdálenost kol s motorem. Ty totiž při otáčení obkreslují kružnici a nás zajímá, jaká je $1/4$ jejího obvodu -- to totiž přesně odpovídá otočení robota o 90\degree.

		Jakmile si toto uvědomíme, tak stačí spočítat čtvrtinu obvodu přes vzoreček, který již známe:
		$$
		\begin{aligned}
			\text{vzdálenost} &= \frac{1}{4} \cdot 2 \cdot \pi \cdot \text{vzdálenost kol} \\
			&= \frac{1}{2} \cdot \pi \cdot \text{vzdálenost kol}
		\end{aligned}
		$$
		Jelikož chceme tuto vzdálenost ujet, tak ji stačí vložit do vzorečku pro počet otáček kol:
		$$
		\begin{aligned}
			\text{otáčky} &= \frac{\text{vzdálenost}}{\text{obvod}} \\
			&=  \frac{\frac{1}{2} \cdot \pi \cdot \text{vzdálenost kol}}{2 \cdot \pi \cdot \text{poloměr}}  \\
			&= \frac{\text{vzdálenost kol}}{4 \cdot \text{poloměr}}
		\end{aligned}
		$$
	\end{solution*}

	\subsection{Výpočet pozice robota}
	K tomu, abychom mohli robotu říct \textit{dojeď na tyto souřadnice} potřebuje robot vědět, kde se aktuálně nachází. To si může počítat z informací o tom, kolik otáček udělaly jeho kolečka (enkódery), a jakým směrem se aktuálně dívá (gyroskop).

	Tímto problémem se zabývá \textbf{odometrie}. TODO


\end{document}
