\documentclass[../main.tex]{subfiles}


\begin{document}
	\section{Základy programování v jazyce Blocky}

	\subsection{Základní ovládání motorů}
	K ovládání motorů robota budeme pro začátek používat dva základní příkazy (k nalezení v příkazové části, v sekci \textbf{VEX V5 Motors}):
	\begin{itemize}
		\item[\block] \blockimage{2em}{../Images/02/motor-start.png} -- zapne daný motor.
		\item[\block] \blockimage{1.5em}{../Images/02/motor-stop.png} -- vypne daný motor.
	\end{itemize}

	Když robot vykonává náš program a potká tyto dva bloky, tak zapne/vypne daný motor a jde okamžitě na další příkaz. Takovým blokům říkáme \textbf{neblokující}, protože „neblokují“ běh programu.

	K tomu, abychom mohli psát komplexnější programy, budeme potřebovat ještě jeden příkaz, se kterým jsme se setkali v minulé části:
	\begin{itemize}
		\item[\block] \blockimage{2em}{../Images/02/wait.png} -- počká daný čas.
	\end{itemize}

	Tento příkaz je \textbf{blokující} -- jakmile na něho při běhu programu dojde, tak se musí počkat, než uplyne zadaný čas, než se začnou vykonávat bloky pod ním.

	\begin{question}
		Naprogramujte robota, aby jel dvě vteřiny rovně, obrátil se (otáčením doleva/doprava) a poté jel tři vteřiny dozadu.
	\end{question}

	\begin{question}\label{que:loop-example}%
		Naprogramujte robota, aby jel pravou stranou dopředu 1 sekundu a poté levou stranou dopředu 1 sekundu. Opakujte třikrát.
	\end{question}

	\subsection{Smyčky}
	V příkladu \ref{que:loop-example} jsme několikrát opakovali stejný kód, což může být otravné. Výrazně jednodušší by bylo, kdybychom mohli prohlásit „opakuj tento kód.“ To je naštěstí přesně to, co dělají smyčky. Dvě nejzákladnější (k nalezení v příkazové části, v sekci \textbf{Loops}) vypadají takto:
	\begin{itemize}
		\item[\block] \blockimage{4em}{../Images/02/loop.png} -- opakuje bloky uvnitř několikrát.
		\item[\block] \blockimage{4em}{../Images/02/loop-forever.png} -- opakuje bloky uvnitř donekonečna.
	\end{itemize}

	\begin{question}
		Upravte příklad \ref{que:loop-example}, aby využíval smyčku.
	\end{question}

	\begin{question}
		Naprogramujte robota, aby donekonečna jezdil do čtverce (1 vteřinu rovně, poté otočení doprava/doleva a opakovat...).
	\end{question}

	\subsection{Jízda na vzdálenost}
	Zatím umíme pouze zapnout daný motor a čekat nějakou dobu, ale občas by se nám hodilo říct motoru „otoč se o tolik stupňů/otáček.“ Kombinací příkazů motoru a smyček toho můžeme docílit. Budeme potřebovat následující nové bloky:
	\begin{itemize}
		\item[\block] \blockimage{2em}{../Images/02/motor-distance-start.png} -- zapne motor, dokud se neotočí o daný počet stupňů (\textit{degrees}) nebo otáček (\textit{revolutions}). Je opět neblokující -- zapne motor a jde na další block.
		\item[\block] \blockimage{4em}{../Images/02/loop-while.png} -- \parbox{0.725\textwidth}{opakuje bloky uvnitř, dokud (\textit{while}) nebo než (\textit{until}) platí podmínka. Pokud uvnitř nejsou žádné bloky, tak pouze blokuje, dokud/než podmínka platí}.
		\item[\block] \blockimage{1.5em}{../Images/02/motor-done.png} -- podmínka, zda daný motor skončil; k nalezení v sekci \textbf{VEX V5 Motors}.
	\end{itemize}

	Možná se zdá, ze by stačilo za \tcbox[colback=ceruleanblue,coltext=white]{start} blok dát první blok v seznamu výše -- to ale nebude stačit, protože jakmile program dojde za poslední blok, tak vše skončí a robot se zastaví. Až teprve smyčkou mu řekneme, že má počkat.

	\begin{question}
		Naprogramujte robota, aby jel rovně pět otáček, obrátil se (otáčením doleva/doprava, opět pomocí otáček) a poté jel dvanáct otáček dozadu.
	\end{question}

\end{document}
