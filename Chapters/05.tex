\documentclass[../main.tex]{subfiles}


\begin{document}
	\section{Pokročilejší programování v jazyce Blocky}

	\subsection{Funkce}
	Při programování složitějších programů v Blocky (a ostatních jazycích) se občas hodí mít „zkratky“ pro části kódu, které často používáme.

	Kdybychom měli například posloupnosti příkazů pro otáčení doleva, otáčení doprava a jízdu rovně, a chtěli bychom naprogramovat robota pro jízdu v komplikovaném bludišti, tak není moc příjemné kopírovat stejných několik příkazů za sebou.

	Právě k tomu se hodí funkce. K jejich vytváření a používání budeme používat dva bloky (k nalezení v části \textbf{Functions}):
	\begin{itemize}
		\blockFunctionDefinition
		\blockFunctionCall
	\end{itemize}

	\begin{question*}
		Vytvořte tancujícího robota -- definujte funkce \texttt{doprava}, \texttt{doleva}, \texttt{dopředu} a \texttt{dozadu}, které posunou robota o malý kousek daným směrem. Poté vezměte svou oblíbenou písničku a pomocí volání funkcí naprogramujte robota, aby do písničky tancoval!
	\end{question*}

	\begin{question*}
		Vytvořte bludiště, které robot musí projet (kraje jde vymezit např. volnými kovovými díly stavebnice). K otáčení doprava/doleva použijte funkce.
	\end{question*}

	\subsection{Proměnné}

	\subsubsection{Čísla}

	\subsubsection{Text}

	\subsection{Dálkové ovládání}
\end{document}
