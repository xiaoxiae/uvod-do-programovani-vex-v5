\documentclass[../main.tex]{subfiles}


\begin{document}
	\section{Pokročilejší programování v jazyce Blocky}

	\subsection{Funkce}
	Při programování složitějších programů v Blocky (a ostatních jazycích) se občas hodí mít „zkratky“ pro části kódu, které často používáme.

	Kdybychom měli například posloupnosti příkazů pro otáčení doleva, otáčení doprava a jízdu rovně, a chtěli bychom naprogramovat robota pro jízdu v komplikovaném bludišti, tak není moc příjemné kopírovat stejných několik příkazů za sebou.

	Právě k tomu se hodí funkce. K jejich vytváření a používání budeme používat dva bloky (k nalezení v části \textbf{Functions}):
	\begin{itemize}
		\blockFunctionDefinition
		\blockFunctionCall
	\end{itemize}

	\begin{question}
		Vytvořte tancujícího robota -- definujte funkce \texttt{doprava}, \texttt{doleva}, \texttt{dopředu} a \texttt{dozadu}, které posunou robota o malý kousek daným směrem. Poté použijte svou oblíbenou písničku a pomocí volání funkcí robota naprogramujte, aby do písničky tancoval!
	\end{question}

	\begin{question*}
		Vytvořte bludiště, které robot musí projet (kraje jde vymezit např. volnými kovovými díly stavebnice). K otáčení doprava/doleva použijte funkce.
	\end{question*}

	\subsection{Proměnné}
	Proměnné jsou krabičky, do kterých si robot může při běhu programu ukládat informace různých druhů, jako např. čísla nebo slova. Jsou jedním z nejdůležitějších konceptů v programování ve všech jazycích a dovolují nám dělat řadu věcí, které bychom s našimi dosavadními znalostmi nemohli.

	K vytvoření proměnné stačí jít do sekce \textbf{Variables} v příkazové části, kliknout na \centerimage{\baselineskip}{Images/blocks/variable-create.png}, nějak jí pojmenovat a kliknout na \textbf{OK}. Jako jméno je vhodné zvolit něco, co odpovídá obsahu dané proměnné -- proměnná ukládající počet nárazů robota by se rozhodně neměla jmenovat \texttt{mrkvový\_guláš}!

	Po vytvoření libovolné proměnné budeme mít k dispozici příkazy, kterými s proměnnou budeme moci pracovat (k nalezení v části \textbf{Variables}):
	\begin{itemize}
		\blockVariableChange
		\blockVariableGet
		\blockVariableSet
	\end{itemize}

	 Příklad využití proměnných je např. tento program, který robota otáčí doleva o $1$ otáčku, poté o $2$, dále o $3$, apod. (s tím, že po jízdě vždy chvíli čeká):

	\begin{figure}[h!]%TODO: pozice
		\centering
		\begin{minipage}{0.5\textwidth}
			\includegraphics[width=\linewidth]{Images/05/variable-program.png}
		\end{minipage}
	\end{figure}

	Jména proměnných \texttt{i}, \texttt{j} a \texttt{k} jsou většinou používány pro to, když potřebujeme postupně procházet přes čísla $1, 2, \ldots$ (což právě v tomto příkladu potřebujeme).

	\subsubsection{Čísla}
	Čísla jsou jeden z druhů informací, které do proměnné můžeme ukládat. Hodí se např. když chceme počítat, kolikrát jsme narazili, o kolik stupňů jsme se otočili nebo kolik ještě otáček chceme dělat. K tomu, abychom s nimi mohli pracovat, se nám budou hodit následující bloky (k nalezení v části \textbf{Math}):
  
	\begin{itemize}
		\blockMathOperation
		\blockMathTest
		\blockMathValue
		\blockMathConstant
		\blockMathRandom
	\end{itemize}

	\begin{question}
		Naprogramujte robota, aby jel otáčku dopředu a dozadu, poté dvě otáčky dopředu a dozadu, apod. (donekonečna).
	\end{question}

	\begin{question*}
		Naprogramujte robota, aby jel do trojúhelníku, poté do čtverce, poté do pětistěnu, apod. (donekonečna).
	\end{question*}

	\begin{question}
		Naprogramujte robota, aby ujel vzdálenost, kterou mu na začátku programu uložíte do proměnné \texttt{vzdalenost} (v metrech). Všechny výsledky ke spočítání otáček/stupňů proveďte přes matematické bloky.
	\end{question}

	\begin{question}
		Vytvořte bláznivého robota, který se na místě bude otáčet střídavě doprava/doleva o náhodný úhel.
	\end{question}
	
	\subsubsection{Pravdivostní hodnoty}
	Dalším druhem hodnot, které do proměnných můžeme ukládat, jsou pravdivostní hodnoty (anglicky \textit{boolean values}), které jsou buď pravda (\blockLogicTrueImageBaseline) nebo nepravda (\blockLogicTrueImageBaseline), obě k nalezení v části \textbf{Logic}. Hodí se k ukládání informace o tom, zda je/není něco pravda. Budou důležité v kapitole \ref{cha:logic}

	\subsubsection{Text}
	Posledním druhem hodnot, se kterými se seznámíme, je text. Opět zde nepůjdeme příliš do detailu -- stačí nám vědět, že i text do proměnných ukládat umíme a můžeme s ním různými operacemi pracovat (k nalezení v částech \textbf{Text} a \textbf{VEX V5 Display}):

	\begin{itemize}
		\blockString
		\blockDisplayPrint
		\blockDisplayClear
	\end{itemize}

	\begin{question}
		Naprogramujte robota, aby jel donekonečna rovně a ukazoval, kolik otáček již ujel.
	\end{question}

	\subsection{Podmínky}\label{cha:logic}

	\subsection{Dálkové ovládání}

\end{document}
