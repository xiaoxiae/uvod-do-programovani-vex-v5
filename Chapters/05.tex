\documentclass[../main.tex]{subfiles}


\begin{document}
	\section{Pokročilejší programování v jazyce Blocky}

	\subsection{Funkce}
	Při programování složitějších programů v Blocky (a ostatních jazycích) se občas hodí mít „zkratky“ pro části kódu, které často používáme.

	Kdybychom měli například posloupnosti příkazů pro otáčení doleva, otáčení doprava a jízdu rovně, a chtěli bychom naprogramovat robota pro jízdu v komplikovaném bludišti, tak není moc příjemné kopírovat stejných několik příkazů za sebou.

	Právě k tomu se hodí funkce. K jejich vytváření a používání budeme používat dva bloky (k nalezení v části \textbf{Functions}):
	\begin{itemize}
		\blockFunctionDefinition
		\blockFunctionCall
	\end{itemize}

	\begin{question}
		Vytvořte tancujícího robota -- definujte funkce \texttt{doprava}, \texttt{doleva}, \texttt{dopředu} a \texttt{dozadu}, které posunou robota o malý kousek daným směrem. Poté použijte svou oblíbenou písničku a pomocí volání funkcí robota naprogramujte, aby do písničky tancoval!
	\end{question}

	\begin{question*}
		Vytvořte bludiště, které robot musí projet (kraje jde vymezit např. volnými kovovými díly stavebnice). K otáčení doprava/doleva použijte funkce.
	\end{question*}

	\subsection{Proměnné}
	Proměnné jsou krabičky, do kterých si robot může při běhu programu ukládat informace různých druhů, jako např. čísla nebo slova. Jsou jedním z nejdůležitějších konceptů v programování ve všech jazycích a dovolují nám dělat řadu věcí, které bychom s našimi dosavadními znalostmi nemohli.

	K vytvoření proměnné stačí jít do sekce \textbf{Variables} v příkazové části, kliknout na \centerimage{\baselineskip}{../Images/blocks/variable-create.png}, nějak jí pojmenovat a kliknout na \textbf{OK}. Jako jméno je vhodné zvolit něco, co odpovídá obsahu dané proměnné -- proměnná ukládající počet nárazů robota by se neměla jmenovat \texttt{mrkvový\_guláš}!

	Po vytvoření libovolné proměnné budeme mít k dispozici příkazy, kterými s proměnnou budeme moci pracovat:
	\begin{itemize}
		\blockVariableChange
		\blockVariableGet
		\blockVariableSet
	\end{itemize}

	Příklad použití proměnných v programu je např. pokud bychom chtěli napsat program, který otočí robota doprava o 90 stupňů potom 180 TODO dopsat tenhle problém.

	\subsubsection{Čísla a matematika}

	\begin{question}
		Naprogramujte robota, aby jel otáčku dopředu a dozadu, poté dvě otáčky dopředu a dozadu, apod. (donekonečna).
	\end{question}

	\begin{question*}
		Naprogramujte robota, aby jel do trojúhelníku, poté do čtverce, poté do pětistěnu, apod. (donekonečna).
	\end{question*}

	\subsubsection{Text}
	TODO: a jeho zobrazování!

	\subsection{Dálkové ovládání}
\end{document}
