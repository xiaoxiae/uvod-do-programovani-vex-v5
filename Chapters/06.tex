\documentclass[../main.tex]{subfiles}


\begin{document}
	\section{Základy programování v jazyce Python}
	Až doposud jsme programovali v jazyce Blocky, protože je příjemný na používání a pochopení. Může ale být v některých věcech dost omezující. Kdybychom v něm chtěli programovat pokročilejší programy (nad rámec této učebnice), tak by se nám to dělalo velice špatně.

	Naštěstí jsme programováním v jazyce Blocky vlastně programovali také v Pythonu, protože Blocky je vlastně „nadstavba“ Pythonu, aby se s ním hezky a interaktivně pracovalo. To je skvělé, protože díky tomu je snadnější se Python naučit studováním toho, jak vypadají naše Blocky programy převedené do Pythonu.

	Zkusme se podívat, jak vypadá jeden z našich základních programů v Blocky, převedený do Pythonu. Jedná se o úplně první program, který řekne robotu, aby jel tři vteřiny rovně.

	\begin{minted}{python}
		import vex
		import drivetrain
		import smartdrive
		import sys
		
		#region config
		brain    = vex.Brain()
		motor_1  = vex.Motor(vex.Ports.PORT1, vex.GearSetting.RATIO18_1, False)
		motor_10 = vex.Motor(vex.Ports.PORT10, vex.GearSetting.RATIO18_1, False)
		#endregion config
		
		# main thread
		motor_1.spin(vex.FORWARD, 100, vex.PERCENT)
		motor_10.spin(vex.FORWARD, 100, vex.PERCENT)
		vex.wait(3,vex.SECONDS)
		motor_1.stop(vex.BrakeType.HOLD)
		motor_10.stop(vex.BrakeType.HOLD)	
	\end{minted}

	Pojďme si program rozebrat po částech.

	\begin{minted}{python}
		import vex
		import drivetrain
		import smartdrive
		import sys
	\end{minted}

	TODO


	\subsection{Proměnné}

	\subsection{Smyčky}

	\subsection{Podmínky}

\end{document}
